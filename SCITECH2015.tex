\documentclass[english]{aiaa-tc}
\usepackage{lmodern}
\renewcommand{\sfdefault}{lmss}
\renewcommand{\ttdefault}{lmtt}
\usepackage[T1]{fontenc}
\usepackage[latin9]{inputenc}
\usepackage{array}
\usepackage{multirow}
\usepackage{amssymb}
\usepackage{graphicx}
%\usepackage{nomencl}
% the following is useful when we have the old nomencl.sty package
%\providecommand{\printnomenclature}{\printglossary}
%\providecommand{\makenomenclature}{\makeglossary}
%\makenomenclature
\makeatletter

%%%%%%%%%%%%%%%%%%%%%%%%%%%%%% LyX specific LaTeX commands.
%% Because html converters don't know tabularnewline
\providecommand{\tabularnewline}{\\}
%% A simple dot to overcome graphicx limitations
\newcommand{\lyxdot}{.}


%%%%%%%%%%%%%%%%%%%%%%%%%%%%%% User specified LaTeX commands.
\usepackage{wrapfig}% embedding figures/tables in text (i.e., Galileo style)
 \usepackage{threeparttable}% tables with footnotes
 \usepackage{dcolumn}% decimal-aligned tabular math columns
  \newcolumntype{d}{D{.}{.}{-1}}
 %\usepackage{nomencl}% automatic nomenclature generation via makeindex
  \makeglossary
 \usepackage{subfig}% subcaptions for subfigures
 \usepackage{fancyvrb}% extended verbatim environments
  \fvset{fontsize=\footnotesize,xleftmargin=2em}
 \usepackage{lettrine}% dropped capital at beginning of paragraph
% \usepackage[dvips]{dropping}% alternative dropped capital package
% \usepackage{hyperref}% embedding hyperlinks 
% \usepackage{morefloats}
 % define some commands to maintain consistency
 \newcommand{\pkg}[1]{\texttt{#1}}
 \newcommand{\cls}[1]{\textsf{#1}}
 \newcommand{\file}[1]{\texttt{#1}}

\@ifundefined{showcaptionsetup}{}{%
 \PassOptionsToPackage{caption=false}{subfig}}
\usepackage{subfig}
\makeatother

\usepackage{babel}
\begin{document}
\title{A Study of the Noise Source Mechanisms in an Excited Mach 0.9 Jet - Complementary Experimental and Computational Analysis}


\author{Michael Crawley\thanks{Graduate Research Assistant. Student Member, AIAA}, \
Rachelle L. Speth\thanks{Graduate Research Assistant. Student Member, AIAA},
 Datta V. Gaitonde\thanks{John Glenn Chair Professor. Fellow, AIAA},
 and Mo Samimy\thanks{John B. Nordholt Professor. Fellow, AIAA}
\\\normalsize\itshape Mechanical and Aerospace Engineering, The Ohio State University, Columbus, OH \\}


\maketitle

\begin{abstract}
	The abstract goes here...
\end{abstract}

\section{Introduction}

\section{Experimental Setup}
Experimentation was conducted in the free jet facility (a schematic of which can be found in Fig. \ref{GDTLschematic}) at the GDTL within the Ohio State University's Aerospace Research Center.The dimensions of the chamber are 5.14 m wide by 4.48 m long and 2.53 m high (wedge-tip to wedge-tip). The design of the chamber produces an anechoic cutoff frequency of 160 Hz, below the frequencies of interest for this study. Additional details of the facility design and validation can be found in Hahn \cite{Hahn2011}. Compressed, dried, and filtered air is supplied by two cylindrical storage tanks with a total capacity of 43 m$^3$ and maximum pressure of 16 MPa; the tanks are charged by three, five-stage reciprocating compressors. The air enters the facility horizontally, passes through a stagnation chamber and turbulence screens, and exhausts through a converging nozzle. Opposite the nozzle, a collector accumulates the jet and entrained air and exhausts to the outdoors.

A converging, axisymmetric nozzle with exit diameter of 25.4 mm was used in the current study. The internal contour of the nozzle was designed using a fifth order polynomial. The nozzle utilized a thick-lipped design in order to simplify the mounts for the LAFPA extension, which housed the eight actuators used in this study.For the experiments reported in this paper, the jet was operated at a Mach number, $M_{j}$, of 0.90, and with a total temperature ratio of unity.The Reynolds number based on the jet exit diameter was $6.2 \times 10^{5}$; previous investigations using hot-wire anemometry have indicated that the initial shear layer is turbulent for this operating condition with momentum thickness ~0.09 mm and boundary layer thickness ~1 mm \cite{kfm2009-1}. 
	\begin{figure}
		\begin{center}
			\includegraphics[width=3.5in]{GDTL_facility_schematic.png}
			\caption{Plan view of GDTL free jet facility; dimensions in meters.}\label{GDTLschematic}
		\end{center}
	\end{figure}

\section{Computational Model}\label{theo}
The simulations employ the same approach as previously used
to simulate a Mach~$1.3$ jet without and with
control\cite{gdv2011-POF,SpethCF2013}.  The full compressible
Navier-Stokes equations are solved in curvilinear coordinates
($\xi,\eta,\zeta$) using the
strong conservative form\cite{vm74-1,sjl78-1}. The transformed
non-dimensional equations in 
vector notation are given as: 
\begin{equation}
\frac{\partial}{\partial\tau}\left(\frac{\vec{U}}{J}\right)+\frac{\partial\hat{F}}{\partial\xi}+\frac{\partial\hat{G}}{\partial\eta}+\frac{\partial\hat{H}}{\partial\zeta}=\frac{1}{Re}\left[\frac{\partial\hat{F}_{v}}{\partial\xi}+\frac{\partial\hat{G}_{v}}{\partial\eta}+\frac{\partial\hat{H}_{v}}{\partial\zeta}\right]\label{navier}
\end{equation}
where $\vec{U}=\left\{ \rho,\rho u,\rho v,\rho w,\rho E\right\} $
denotes the solution vector and
$J=\partial\left(\xi,\eta,\zeta,\tau\right)/\partial\left(x,y,z,t\right)$
is the transformation Jacobian.  Details of the various terms in
Eqn.~\ref{navier} may be found in Speth and 
Gaitonde\cite{speth2012b}.
For the inviscid terms, a third-order upwind biased approach is
adopted, together with the Roe scheme\cite{rpl81-1} for flux evaluation.  The
limiter required to enforce monotonicity is a crucial 
component of the method.  The van Leer harmonic limiter\cite{lbv79-1}
has proven to be very successful at reproducing the main features of the
unsteadiness in the jet.  The viscous 
terms are discretized with second-order centered differences and time
integration is performed by a second-order diagonalized~\cite{pth81-1}
approximately 
factored method~\cite{br78-1}.  A sub-iteration
strategy is used to minimize errors due to factorization, linearization and
explicit boundary condition implementation.
\begin{figure}
\begin{center}
	\includegraphics[width=3.5in]{MACH09ComputationalDomain1.png}
\caption{Computational domain}\label{fig:M09Computationaldomain}
\end{center}
 \end{figure}

 A $65$ million point mesh (Fig.~\ref{fig:M09Computationaldomain}) is
 used to simulate the Mach~$0.9$ jet measured in the experiment (Fig.
 \ref{GDTLsetup}a).  The grid has dimensions of $685$ points on the
 $\xi$ (streamwise) direction, $455$ points in the $\eta$ (radial)
 direction, and $209$ points in the $\zeta$ (azimuthal) direction. In
 the radial direction, the mesh is refined in the nozzle region and
 gradually stretched in the far field. At the exit of the nozzle, the
 grid maintains a constant axial spacing until after the potential
 core length; then stretches in the streamwise direction as well. To
 preserve continuity, the grid has a five point overlap in the $\zeta$
 direction. Characteristic boundary conditions\cite{bj2000-1} are
 applied to the upstream (outside the nozzle) and radial boundaries.
 Non-reflecting conditions are applied to the downstream and far-field
 boundaries. Stagnation conditions are specified at the first $\xi$
 plane of the nozzle ($\rho_{inlet}=2.04kg/m^{3}$, $U_{inlet}=22m/s$, $P_{inlet}=171,427Pa$) to
%???? Why specify rho, U and P at the entrance, but rho U and T at the exit???
 achieve perfectly expanded nozzle exit conditions corresponding to
 $\rho_{jet}=1.404kg/m^{3}$, $U_{jet}=285.99m/s$, $T_{jet}=251.31K$ which match the experiments.
 Based on the nozzle diameter therefore, the Reynolds number is
 $Re=635,308$. The nozzle geometry resembles that of the 
 experiments including the nozzle ring on which the actuators are
 mounted. 
The velocity profile at the entrance to the nozzle is that
 of a uniform flow (zero at the wall and $U_{inlet}$ everywhere else).
 Perturbations were not introduced into the inflow due to the
 unknown perturbations in the experiment. Therefore, the simulations
 have a laminar boundary layer at the nozzle exit while the
 experiments have a very thin turbulent boundary layer (the momentum
 thickness has been estimated to be $0.09 mm$).  Previous studies have
 shown that despite this difference, the main features of the
 experimental observations are successfully reproduced by the
 computations\cite{gdv2011-POF,SpethCF2013}.  Other studies have shown
 that a smaller $32$ million point simulation is adequate to reproduce
 the features of the experiment\cite{spethASME2013}.

\begin{figure}[h]
\begin{center}
\includegraphics[width=3.6in]{actuatormodelnew}
\caption{The computational domain including the nozzle  (a), and the numerical actuator model (b)}\label{fig:actuator}
\end{center}
\end{figure}
The LAFPAs are modeled after the experiments using a surface heating
technique to excite jet shear layer instabilities and azimuthal modes
within the jet.  Eight actuators are placed around the periphery of
the jet on the nozzle collar at the locations and dimensions of the
experiments as explained above in Section~\ref{lafpa}. As shown in
Fig.~\ref{fig:actuator}b, each actuator consists of a heated region of
the nozzle wall which extends the azimuthal length corresponding to
the separation distance between electrodes ($3 mm$) and has an axial
extent equal to the length of the groove ($1 mm$). The temperature of
the nozzle wall was assumed to be $1.12T_{\infty}$.  When the actuator
is on the temperature of the actuator region increases to
$5T_{\infty}$. Little difference was seen in the previous work (Speth
and Gaitonde\cite{SpethASM2012}) for the temperature range measured in
experiments (Utkin {\em et al.}\cite{uyg2007-2}) for a Mach number of
1.3.  The semi-empirical model is necessary to avoid first-principles
simulation of the poorly understood plasma heating process, as well as
to restrict the required computational resources to feasible levels
(see Ref.~\cite{GaitondeCAF2013-1}).

Unlike acoustic drivers, the LAFPAs are on-off devices and thus can be
represented by rectangular pulses with a duty cycle, which allows for
a wide range of operation choices.  Duty cycle is the percentage of
actuator on time in an excitation cycle. Therefore, a duty cycle
of $100\%$ results in the actuator being on all the time.  
The experimental duty cycle varies with frequency, since the arc
strike lasts a fixed time.  Since the actuator
model is empirical, the computational duty cycle was chosen to obtain
similar control authority as in the experiment.  This necessitates a
higher
duty cycle ($10\%$) than the one used in the experiments
($2.0\%$ for $St_{DF}=0.25$).
As noted earlier, despite the simplicity of the model, its success
has been documented in Gaitonde 
and Samimy\cite{gdv2011-POF}, where, in addition to coherent
structures, mean and fluctuating quantities have been compared.
Furthermore, the mean flow structure with control was shown to match
the theoretical predictions of Cohen and Wygnanski~\cite{cj87-2}.

Like the experiments, the axisymmetric ($m=0$) mode was employed to
study a range of Strouhal numbers. The Strouhal numbers studied in the
simulations include: $0.05$, $0.15$, and $0.25$. Data was acquired
every timestep at the point probes depicted in Fig.~\ref{GDTLsetup}b
as well as on several $\xi$, $\eta$, and $\zeta$ computational planes.
Phase-averaged data were also computed for each of the simulations.

%Ani's 2012 physics of fluids paper should be cited here [MC]

%should i try to explain? The lower Reynolds number jet is harder to
%control due to the excitation having to combat the naturally
%occurring organized large scale structures. While in the high
%Reynolds number case the large scale structures are less organized
%(more turbulent) and therefore are easier to control. ------I would
%mention it. [MC] 

\section{Results}\label{results} 


\section*{Acknowledgments}
   Computational resources were provided by the DoD HPCMP (AFRL, NAVO
   and ERDC) and the Ohio Supercomputer Center. The support of this
   complementary experimental and computational work by the Air Force
   Office of Scientific Research (Dr. John Schmisseur and
   Dr. Rengasamy Ponnappan) is greatly appreciated. Several figures
   were made using Fieldview software with licenses obtained from the
   Intelligent Light University Partnership Program. 

\bibliographystyle{aiaa}
\bibliography{./NEWMASTER}
%\printnomenclature

\end{document}
